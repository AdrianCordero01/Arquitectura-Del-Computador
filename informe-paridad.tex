\documentclass[12pt]{article}
\usepackage[spanish]{babel}
\usepackage{array}
\usepackage{xcolor}
%opening
\title{\textbf{{\Huge Informe Practica I}}}
\author{Adrian Cordero}

\begin{document}

\maketitle
\newpage

\section{Introducción}
El presente informe analiza la implementación de una función de paridad en arquitectura MIPS32, comparando los enfoques recursivo e iterativo. Se detalla el uso de la pila, los riesgos asociados y las diferencias con los modelos teóricos.

\section{Implementación de la recursividad y el rol de \$sp}
La recursividad en MIPS se implementa mediante la gestión manual de la memoria. A diferencia de los lenguajes de alto nivel, el hardware no  ''recuerda'' a dónde debe volver tras una función.

\subsection{El Registro \$ra (Return Address):}
Al usar \textbf{jal}, la dirección de retorno se guarda en \textcolor{red}{\textbf{\$ra}}. Si la función se llama a sí misma, ese valor se sobrescribe.

\subsection{El Stack Pointer (\$sp):}
Actúa como un ''ancla'' en la memoria \textbf{RAM}. Para que la recursión funcione, debemos guardar el valor de \textcolor{red}{\textbf{\$ra}} (y cualquier registro \$s o \$t que queramos preservar) en la pila antes de la siguiente llamada.

\subsection{LIFO (Last In, First Out):}
Cada llamada reserva espacio, pila crece hacia abajo,\textbf{ \textcolor{blue}{addi} \textcolor{red}{\textbf{\$ra}}, \textcolor{red}{\textbf{\$ra}}, -4} y cada retorno libera espacio \textbf{ \textcolor{blue}{addi} \textcolor{red}{\textbf{\$ra}}, \textcolor{red}{\textbf{\$ra}}, 4}

\section{Riesgos de desbordamiento y mitigación}
Existen dos tipos principales en este contexto:
\subsection{Stack Overflow:}
Ocurre si \textbf{n} es muy grande. Cada llamada recursiva consume 4 bytes (o más si guardas más registros). Si la recursión es muy profunda, la pila choca con los datos globales o el heap.

\subsubsection{Mitigación:}
Usar implementaciones iterativas o limitar el rango de entrada de\textbf{ n}.

\subsection{Desbordamiento de Enteros:}
Si realizas operaciones aritméticas que superen los 32 bits.

\subsubsection{Mitigación:}
Usar instrucciones como \textbf{addu} (unsigned) si no necesitas atrapar excepciones, o validar rangos antes de operar.

\section{Diferencias: Iterativo vs. Recursivo}

% TODO: \usepackage{array} required
\begin{tabular}{|>{\centering\arraybackslash}m{4cm}|>{\centering\arraybackslash}m{4cm}|>{\centering\arraybackslash}m{4cm}|}
	\hline
\textbf{	Característica} & \textbf{Implementación Iterativa} & \textbf{Implementación Recursiva} \\
	\hline
\textbf{	Memoria (Pila)} & No usa \textcolor{red}{\$sp} & Si usa \textcolor{red}{\$sp} \\
	\hline
\textbf{	Registros} & \textcolor{red}{\$a0}, \textcolor{red}{\$v1}, \textcolor{red}{\$t0},  \textcolor{red}{\$ra} & \textcolor{red}{\$a0}, \textcolor{red}{\$v1}, \textcolor{red}{\$t0},  \textcolor{red}{\$ra} que guarda/restaura contentmente \\
	\hline
	\textbf{Velocidad} & \textit{\textbf{O(n)}} solo con saltos \textcolor{blue}{j} & \textit{\textbf{O(n)}} con accesos a memoria \textcolor{blue}{sw}/\textcolor{blue}{lw} \\
	\hline
\end{tabular}

\section{Ejemplos académicos vs. Código operativo}
Los libros (como Patterson \& Hennessy) suelen mostrar fragmentos aislados para explicar un concepto. Un código operativo en MARS requiere:

\subsection{Directivas de datos y texto:}
\textbf{\textcolor{magenta}{\textit{.data y .text.}}}

\subsection{Punto de entrada:}
La etiqueta\textbf{ main}: y la llamada al sistema para finalizar el programa:\\ \begin{center}
	\textbf{\textcolor{blue}{li} \textcolor{red}{\$v0}, 10 }
\end{center}

\subsection{Preservación de registros:}
Los ejemplos de libros a veces omiten el guardado de registros temporales para mayor claridad, pero en un programa real, esto causaría errores de lógica

\section{Tutorial paso a paso en MARS}
Para probar tu función paridad:

\subsection{Ensamblar:}
Haz clic en el icono de la llave inglesa y el destornillador. Revisar que no haya errores en la consola inferior.

\subsection{Preparar Registros:}
En la pestaña "Registers", busca \$a0 y dale un valor (ej. 5).

\subsection{Ejecución Paso a Paso:}
Usa el icono de la flecha \textcolor{green}{\textbf{verde}} con un ''\textbf{1}''.

\begin{itemize}
	\item Observar como \textcolor{red}{\textbf{\$sp}} disminuye en cada \textcolor{blue}{\textbf{jal}}.
	\item Mirar la pestaña "Data Segment" para ver cómo se guarda el
		valor de \$ra en la memoria.
		
	\item Sigue el flujo hasta que \textcolor{red}{\textbf{\$a0}} llegue a \textbf{0} y comience el "desenrollado" de la pila.
	
\end{itemize}

\subsection{Verificar Resultado:}
Al terminar, el resultado final estará en \textcolor{red}{\textbf{\$v1}}.

\section{Justificación del enfoque}
Para MIPS, el enfoque iterativo es superior.

\subsection{Eficiencia:}
El procesador MIPS está diseñado para ejecutar instrucciones simples rápido. El acceso a memoria (sw/lw) es costoso en ciclos de reloj.

\subsection{Claridad:}
Aunque la recursión es elegante matemáticamente, en ensamblador se vuelve difícil de leer por la gestión constante de la pila. La versión iterativa es un bucle limpio de 5 líneas.

\section{Análisis y Discusión de los Resultados}
\begin{itemize}
	\item  La versión recursiva de paridad(100) hará que el puntero de pila \$sp se mueva 100 veces, consumiendo 400 bytes.
	\item  La versión iterativa hará las mismas 100 vueltas pero sin tocar la memoria RAM, manteniendo el rendimiento al máximo.
	\item  Conclusión: La recursividad en ensamblador debe reservarse para algoritmos donde la estructura de datos sea inherentemente recursiva (como árboles), pero para paridad o factoriales, la iteración es la elección técnica correcta.
	
\end{itemize}


\end{document}
